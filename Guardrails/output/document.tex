\documentclass[a4paper,12pt]{article}
\usepackage[utf8]{inputenc}
\usepackage{geometry}
\usepackage{longtable}
\usepackage{graphicx}
\usepackage{titlesec}
\usepackage{enumitem}
\usepackage{fancyhdr}
\usepackage{lastpage}
\geometry{margin=1in}

% Configuração do cabeçalho e rodapé
\pagestyle{fancy}
\fancyhf{}
\fancyhead[L]{Procedimento Operacional Padrão (POP)}
\fancyhead[R]{\leftmark}
\fancyfoot[C]{Página \thepage\ de \pageref{LastPage}}

% Configuração de títulos e seções
\titleformat{\section}{\bfseries\large}{\thesection.}{1em}{}
\titleformat{\subsection}{\bfseries\normalsize}{\thesubsection.}{1em}{}
\setlist[itemize]{label=--}

% Título do Documento
\title{
    \vspace{-2cm}
    \rule{\linewidth}{0.5mm}\\[0.4cm]
    \textbf{Procedimento Operacional Padrão (POP)}\\[0.2cm]
    \textbf{Manutenção Impressora Bematech MP 4200 TH}\\[0.2cm]
    \rule{\linewidth}{0.5mm}
}
\author{
    \textbf{Setor Responsável:} TI \\
    \textbf{Elaborado por:} Engenheiro de Produção \\
    \textbf{Data:} \today
}
\date{}

\begin{document}

\maketitle
\vspace{-1cm}

% 1. Objetivo
\section*{1. Objetivo}
Descrever o procedimento para garantir o funcionamento adequado da impressora Bematech MP 4200 TH, verificando a conexão elétrica.

% 2. Aplicação
\section*{2. Aplicação}
Este POP se aplica a todos os usuários que necessitem solucionar problemas básicos de funcionamento da impressora Bematech MP 4200 TH.

% 3. Materiais, Ferramentas e Condições Necessárias
\section*{3. Materiais, Ferramentas e Condições Necessárias}
\begin{itemize}
    \item Impressora Bematech MP 4200 TH
    \item Cabo de força
    \item Conector de alimentação
    \item Tomada elétrica funcionando
\end{itemize}

% 4. Procedimento - Passo a Passo
\section*{4. Procedimento - Passo a Passo}
\begin{enumerate}
    \item \textbf{Passo 1:} Verifique se a impressora Bematech MP 4200 TH está ligada na tomada elétrica.
    \item \textbf{Passo 2:} Examine o cabo de força e o conector de alimentação, certificando-se de que estão firmemente conectados à impressora e à tomada elétrica.
    \item \textbf{Passo 3:} Observe se há algum sinal de dano ou desgaste nos cabos. Caso positivo, reporte ao suporte técnico.
    \item \textbf{Passo 4:} Após conectar a impressora verifique se a mesma liga normalmente e inicia os procedimentos de inicialização. Caso a impressora não ligue, procure o suporte técnico.
\end{enumerate}

% 5. Indicadores e Métricas de Qualidade
\section*{5. Indicadores e Métricas de Qualidade}
\begin{itemize}
    \item \textbf{Indicador 1:} Impressora ligada e funcionando corretamente.
    \item \textbf{Métricas:}  Ausência de mensagens de erro na tela da impressora e funcionamento normal após a verificação da conexão elétrica.
\end{itemize}

% 6. Observações e Cuidados Especiais
\section*{6. Observações e Cuidados Especiais}
Certifique-se que a tomada elétrica está funcionando corretamente.  Se o problema persistir após seguir este POP, entre em contato com o suporte técnico.

% 7. Revisão e Aprovação
\section*{7. Revisão e Aprovação}
\begin{tabbing}
    \hspace{5cm} \= \textbf{Revisado por:} \hspace{2cm} \= ....................................... \\
    \hspace{5cm} \> \textbf{Aprovado por:} \> ....................................... \\
\end{tabbing}

\end{document}